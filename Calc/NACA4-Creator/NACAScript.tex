%!TEX TS-program = xelatex
%!TEX encoding = UTF-8 Unicode

\documentclass[a4paper,twocolumn]{article}

\usepackage[ngerman]{babel}

\usepackage{graphicx}

\usepackage{verbatim}
\usepackage{listings}
\lstset{language=Java,basicstyle=\ttfamily,
	breaklines=true,numbers=none,showlines=true,frame=single}

\usepackage{fontspec,xltxtra,xunicode}
\defaultfontfeatures{Mapping=tex-text}


\title{NACA-Profil vernetzten mit Gmsh.Script}
\author{Malte Hoffmann}
\date{\today{}}

\begin{document}

\maketitle{}

\clearpage

\section{Benutzen des NACA-Scriptes}

\subsection{Laden und Ausführen}
Um das Script auszuführen, wird Gmsh mit dem Script als Parameter gestartet: \textit{gmsh NACA.script}. Nach dem Laden von Gmsh wird der Benutzer
zu fünf Eingaben aufgefordert:
\begin{enumerate}
\item Die 1. Ziffer der NACA-4er Reihe (Profilwölbung in Prozent)
\item Die 2. Ziffer der NACA-4er Reihe (Zehntel der Wölbungsrücklage)
\item Die 3. Ziffer der NACA-4er Reihe (Zehntel der Profildicke)
\item Die 4. Ziffer der NACA-4er Reihe (Hunderstel der Profildicke)
\item Die Feinheit des Netzes. \begin{itemize}
				\item 1.5 $\approx$ 5500-6000 Gitterzellen (Grob)
				\item 0.7 $\approx$ 9000-10000 Gitterzellen (Mittel)
				\item 0.28 $\approx$ 18000-20000 Gitterzellen (Fein)
				\item 0.15 $\approx$ 30000-33000 Gitterzellen (Super Fein)
			       \end{itemize}
\end{enumerate}
Das Eingabefeld ist vor jeder Eingabe von unzulässigen Zeichen zu bereinigen. In den ersten vier Eingaben ist nur eine Ziffer von 0 bis 9 zulässig.
 Die Eingabe über die Feinheit des Netzes akzeptiert eine beliebige Real-Zahl z.B. 0.687. Bestätigt werden die Eingaben jeweils mit \textit{Enter}.

\subsection{Überprüfen der Eingabe}
Um die Eingabe zu überprüfen, wird die \textit{Message Console} zu Hilfe genommen. Mit \emph{Strg+l} oder über das Menu \emph{Tools} wird sie geöffnet.
 In blauer Schrift ist dort die Information über das erstellte NACA-Profil zu lesen. Stimmt diese, kann das Netz im \emph{Mesh} Modus mit einem Klick 
auf \emph{2D} erstellt werden. Wird das Netz angezeigt, kann das \textit{Statistic} Fenster mit \emph{Strg+i} oder über das Menu \emph{Tools} geöffnet werden.
 Die Anzahl der gewünschten Gitterzellen kann mit der unter \textit{Triangles} verglichen werden.

\subsection{Falsche Eingabe oder ungewünschte Gitterzellenanzahl}
Das Script kann im Modus \textit{Geometry} mit \emph{Reload} neu gestartet werden.

\subsection{Richtiges abspeichern und laden}
Stimmen die Eingaben und die Gitterzellenzahl, kann das Netz gespeichert werden. Das Script besitzt, damit es nicht versehentlich durch den Benutzer geändert 
wird, nur Leserechte. Somit müssen die soeben erstellten Geometrie-Daten in einer neuen Datei gespeichert werden. Zum Speichern wird im Kommandofenster-Menu \emph{File} 
der Unterpunkt \emph{Save As...} aufgerufen. In dem sich öffnenden Fenster kann wie gewohnt in den gewünschten Ordner gewechselt werden. Da Gmsh beim späteren Öffnen 
ein Bug haben kann, sollte der Ordner mit \textit{Favorites} und \textit{Add to Favorites} in den Favoriten gespeichert werden.\\
Um eine neue Datei anzulegen, wird am Ende der Pfadeingabe im Eingabefeld \emph{Filename} der gewünschte Name mit der Endung \textit{.geo} hinzugefügt und mit \emph{OK} 
bestätigt. Der folgende Hacken bei \textit{Save physical group labels} wird entfernt und mit \emph{OK} bestätigt. Ist die Datei schon vorhanden, genügt ein Doppelklick auf diese und mit \emph{Replace} wird sie überschrieben. Auch hier wird der Hacken entfernt und mit \emph{OK} bestätigt.\\
Um die soeben gespeicherte Datei zu öffnen, wird über das Kommandofenster-Menu \emph{File} und \emph{Open...} das Dateifenster geöffnet. Ist dieses leer, kann mit einem 
Klick auf \textit{Favorites} und dem gewünschten Ordner der Inhalt wieder angezeigt werden. Ist die richtige Datei gefunden, reicht ein Doppelklick um sie zu öffnen. 
Wenn alles funktioniert, steht der Pfad inklusive des Dateinamens im oberen Balken des Editorfensters.
\subsection{Speichern des Netzes}
Um das Netz zu speichern, muss zuerst nochmals im Modus \emph{Mesh} auf \emph{2D} geklickt werden. Desweiteren reicht ein Klick auf \emph{Save} um das Netz
mit dem vorher angelegtem Dateinamen und der Endung \textit{.msh} zu speichern. Die Netzdatei wird im gleichen Ordner wie die geöffnete Geometrie-Datei abgelegt.



\end{document}
